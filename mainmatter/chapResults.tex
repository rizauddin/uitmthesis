\chapter{RESULTS AND DISCUSSIONS}
\label{ch:results}

\section{Background of Study}

Suppose that we have a line with equation $y = 2x + 3$. This line cuts the $y$-axis at $y = 3$. We can find the gradient of the line using \eqref{eq:gradient}. Equation~\eqref{eq:tada}, $\ldots$.

\begin{equation} \label{eq:gradient}
    m = \frac{y_2 - y_1}{x_2 - x_1}
\end{equation}

\begin{equation} \label{eq:tada}
e^{\pi i} + 1 = 0
\end{equation}

\subsection{Comparison of Method A with Other Studies}

\begin{table}[ht]
    \caption{Length Units}
    %\begin{tabular}{cc}
    \begin{tabular}{>{\centering\arraybackslash}p{.47\textwidth} >{\centering\arraybackslash}p{.47\textwidth}}
        \toprule %header
        \textbf{Millimeters} & \textbf{Centimeters}\\
        \textbf{mm}          &   \textbf{cm}\\
        \midrule
        1           &   0.1\\ \hline
        10          &   1\\ \hline
        100         &   10\\ \hline
        1000        &   100\\ \hline
        10000       &   1000\\
        \bottomrule
    \end{tabular}
    \par\raggedright Note: This table is useful for $\ldots$.
    \label{table:lengthunits}
\end{table}

\subsubsection{Method A Improved}

Figure~\ref{fig:logouitm} is $\dots$. \lipsum[1-2]

\begin{figure}[ht]
    \centering
    \fbox{ % add box arounf image
        \includegraphics[width=.9\linewidth]{logouitm} % scale=.5 <- 50% of original size
    }
    \caption[Short version for LoF]{Logo UiTM Logo UiTM Logo UiTM Logo UiTM Logo UiTM Logo UiTM Logo UiTM Logo UiTM}
    \label{fig:logouitm}

    \par\raggedright
    Notes/Sources: Phasellus in dui mi. Suspendisse placerat nisl et elit tristique, non congue elit bibendum. Donec mauris libero, vehicula in feugiat vitae.
\end{figure}

\lipsum[2-3]

%% An example of a long table
\begin{longtable}{c|c|c} % Need to have 3 c's if there are 3 columns
\caption{A long table.} \label{tab:alongtable} \\

% header for 1st page
\hline\multicolumn{1}{c|}{\textbf{First column}} & \multicolumn{1}{c|}{\textbf{Second column}} & \multicolumn{1}{c}{\textbf{Third column}} \\ \hline 
\endfirsthead

% What to say at the beginning of the table on the next page, if more than 1 pages.
% Change 3 to the number of columns.
% Comment out these two lines if not needed.
\multicolumn{3}{c}%
{{\bfseries \tablename\ \thetable{} -- continued from previous page}} \\

% header if more than 1 pages.
\hline \multicolumn{1}{c|}{\textbf{First column}} & \multicolumn{1}{c|}{\textbf{Second column}} & \multicolumn{1}{c}{\textbf{Third column}} \\ \hline

%DO NOT REMOVE THIS LINE
\endhead

% What to say at the end of the table, if more than 1 pages.
% Change 3 to the number of columns.
% Comment out this line if not needed.
\hline \multicolumn{3}{|r|}{{Continued on next page}} \\ \hline

%DO NOT REMOVE THIS LINE
\endfoot

% repeat \hline if require more than one line at the end of the table
\hline
\endlastfoot

One & Two & 10.2345667890122 \\ \hline
One & Two & 10.2345667890122 \\ \hline
One & Two & 10.2345667890122 \\ \hline
One & Two & 10.2345667890122 \\ \hline
One & Two & 10.2345667890122 \\ \hline
One & Two & 10.2345667890122 \\ \hline
One & Two & 10.2345667890122 \\ \hline
One & Two & 10.2345667890122 \\ \hline
One & Two & 10.2345667890122 \\ \hline
One & Two & 10.2345667890122 \\ \hline
One & Two & 10.2345667890122 \\ \hline
One & Two & 10.2345667890122 \\ \hline
One & Two & 10.2345667890122 \\ \hline
One & Two & 10.2345667890122 \\ \hline
One & Two & 10.2345667890122 \\ \hline
One & Two & 10.2345667890122 \\ \hline
One & Two & 10.2345667890122 \\ \hline
One & Two & 10.2345667890122 \\ \hline
One & Two & 10.2345667890122 \\ \hline
One & Two & 10.2345667890122 \\ \hline
One & Two & 10.2345667890122 \\ \hline
One & Two & 10.2345667890122 \\ \hline
One & Two & 10.2345667890122 \\ \hline
One & Two & 10.2345667890122 \\ \hline
One & Two & 10.2345667890122 \\ \hline
One & Two & 10.2345667890122 \\ \hline
One & Two & 10.2345667890122 \\ \hline
One & Two & 10.2345667890122 \\
\end{longtable}

\lipsum[1]

%% An example of a long table + landscape orientation
\begin{landscape}

\begin{longtable}{c|c|c|c|c|c} % Need to have 6 c's if there are 6 columns
\caption{A long table.} \label{tab:alongtablelandscape} \\

% header for 1st page
\hline\multicolumn{1}{c|}{\textbf{First column}} &
 \multicolumn{1}{c|}{\textbf{Second column}} &
 \multicolumn{1}{c|}{\textbf{Third column}} &
 \multicolumn{1}{c|}{\textbf{Forth column}} & 
 \multicolumn{1}{c|}{\textbf{Fifth column}} & 
 \multicolumn{1}{c}{\textbf{Last column}}
 \\ \hline 
\endfirsthead

% What to say at the beginning of the table on the next page, if more than 1 pages.
% Change 6 to the number of columns.
% Comment out these two lines if not needed.
\multicolumn{6}{c}%
{{\bfseries \tablename\ \thetable{} -- continued from previous page}} \\

% header if more than 1 pages.
\hline \multicolumn{1}{c|}{\textbf{First column}} & \multicolumn{1}{c|}{\textbf{Second column}} & \multicolumn{1}{c}{\textbf{Third column}} \\ \hline

%DO NOT REMOVE THIS LINE
\endhead

% What to say at the end of the table, if more than 1 pages.
% Change 6 to the number of columns.
% Comment out this line if not needed.
\hline \multicolumn{6}{|r|}{{Continued on next page}} \\ \hline

%DO NOT REMOVE THIS LINE
\endfoot

% repeat \hline if require more than one line at the end of the table
\hline
\endlastfoot

One & Two & 10.2345667890122 & a & 8901223532523 2334235235 & 2334235235 abcdefghijkl 123456\\ \hline
One & Two & 10.2345667890122 & a & 8901223532523 2334235235 & 2334235235 abcdefghijkl 123456\\ \hline
One & Two & 10.2345667890122 & a & 8901223532523 2334235235 & 2334235235 abcdefghijkl 123456\\ \hline
One & Two & 10.2345667890122 & a & 8901223532523 2334235235 & 2334235235 abcdefghijkl 123456\\ \hline
One & Two & 10.2345667890122 & a & 8901223532523 2334235235 & 2334235235 abcdefghijkl 123456\\ \hline
One & Two & 10.2345667890122 & a & 8901223532523 2334235235 & 2334235235 abcdefghijkl 123456\\ \hline
One & Two & 10.2345667890122 & a & 8901223532523 2334235235 & 2334235235 abcdefghijkl 123456\\ \hline
One & Two & 10.2345667890122 & a & 8901223532523 2334235235 & 2334235235 abcdefghijkl 123456\\ \hline
One & Two & 10.2345667890122 & a & 8901223532523 2334235235 & 2334235235 abcdefghijkl 123456\\ \hline
One & Two & 10.2345667890122 & a & 8901223532523 2334235235 & 2334235235 abcdefghijkl 123456\\ \hline
One & Two & 10.2345667890122 & a & 8901223532523 2334235235 & 2334235235 abcdefghijkl 123456\\ \hline
One & Two & 10.2345667890122 & a & 8901223532523 2334235235 & 2334235235 abcdefghijkl 123456\\ \hline
One & Two & 10.2345667890122 & a & 8901223532523 2334235235 & 2334235235 abcdefghijkl 123456\\ \hline
One & Two & 10.2345667890122 & a & 8901223532523 2334235235 & 2334235235 abcdefghijkl 123456\\ \hline
One & Two & 10.2345667890122 & a & 8901223532523 2334235235 & 2334235235 abcdefghijkl 123456\\ \hline
One & Two & 10.2345667890122 & a & 8901223532523 2334235235 & 2334235235 abcdefghijkl 123456\\ \hline
One & Two & 10.2345667890122 & a & 8901223532523 2334235235 & 2334235235 abcdefghijkl 123456\\ \hline
One & Two & 10.2345667890122 & a & 8901223532523 2334235235 & 2334235235 abcdefghijkl 123456\\ \hline
One & Two & 10.2345667890122 & a & 8901223532523 2334235235 & 2334235235 abcdefghijkl 123456\\ \hline
One & Two & 10.2345667890122 & a & 8901222r24234 2334235235 & 2334232235 abcdefghijkl 123456\\
\end{longtable}

\end{landscape}

\lipsum[1]
